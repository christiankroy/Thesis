% !TEX root = /Users/royc/Google_Drive/Thesis/RoyC_Umass_Thesis.tex
\chapter{Appendix - Misc Information} \label{AppendixMisc:Appendix: Misc Info} 
\lhead{Appendix A. \emph{Appendix: Misc Information}} 


\section{Buffers}\label{AppendixMisc:sec:Buffers}

  \renewcommand{\arraystretch}{1}
    \begin{table}[ht]
    \centering
      \begin{tabular}[c]{c|c}
      Component & Concentration \\
        \hline
      Tris-HCl  & 50 mM         \\
        \hline
      MgCl2     & 2 mM          \\
        \hline
      DTT       & 1 mM          \\
        \hline
      ATP       & 400 $\mu$M        \\
        \hline
      pH        & 7.5 @ 25\degree~C   
      \end{tabular}
    \caption[SeqZip Hybridization and Ligation Buffer]
      {
      SeqZip Hybridization and Ligation Buffer
      }
    \label{AppendixMisc:tbl: Rnl2 Buffer}
    \end{table}

\section{Equations}\label{AppendixMisc:sec: Equations}

\subsection{Determining [RNA] from $^{32}$P-$\alpha$-UTP used during vitro transcription}

{
  \tiny{
    \begin{eqnarray*}
    \mu \mbox{M} = \left( \frac{\mbox{pmol}}{\mu\mbox{L}}\right)
          = \left( \frac{\mbox{cpm after purification} \times \mbox{dilution factor}}{\mbox{cpm before purification} \times \mbox{dilution factor}} \right)
          \times \left( \frac{\mbox{mol UTP in original reaction}}{\mbox{Reaction Volume }} \right) \times \left( \frac{1}{\mbox{Number UTPs in transcript}} \right) \times 10^{-12} 
    \end{eqnarray*}
  }
}
%Or you can write the above equation as 

%\begin{eqnarray*}
%\mu \mbox{M}  =  \left( \frac{\mbox{pmol}}{\mu\mbox{L}}\right)
%     & = & \left( \frac{\mbox{cpm after purification} \times \mbox{dilution factor}}{\mbox{cpm before purification} \times \mbox{dilution factor}} \right)\\
%      && \times \left( \frac{\mbox{mol UTP in original reaction}}{\mbox{Reaction Volume}} \right)\\
%      && \times \left( \frac{1}{\mbox{Number UTPs in transcript}} \right) \times 10^{-12} 
%\end{eqnarray*}


\subsection{Determining [RNA] based on A$_{260}$}

  $$
  \mbox{[RNA in M]} = \left( \frac{\mbox{A}_{260} \times \mbox{Dilution Factor}}
                             {10,313 < \mbox{note 1}> \times \mbox{ nucleotides in message}} \right) 
  $$
  
  note 1: This value represents an average RNA extinction ($\epsilon$) coefficient value \\

\subsection{Normalize oxidized small RNA libraries size to time-matched unoxidized library}

NB: this equation assumes calibration against a specific time-point ,
in this case data obtained from 6 week-old testes.

  \begin{eqnarray*}
    \mbox{unox }\tau \mbox{ norm}_1 & = & \left(         
                          \frac{\left( \frac{\displaystyle\sum \mbox{miRNA reads } \tau}{\displaystyle \sum \mbox{miRNA reads 6wk} } \right) \times \mbox{ depth 6wk} }{1,000,000}              
                                            \right)\\
    \mbox{ox }\tau \mbox{ norm}_1 & = &   \mbox{unox }\tau \mbox{ norm}_1 \times
                                   \left(
                                    \frac{\displaystyle \sum \mbox{oxidized shared } \ge \mbox{23 nt reads}}{\displaystyle \sum \mbox{unoxidized shared } \ge \mbox{23 nt reads}}
                                   \right)                                         
    \end{eqnarray*}

\section{PCR Programs}\label{AppendixMisc:sec:PCR Programs}

\textbf{Ligamer Hybridization}
ROY-H2 | Ligamer Hybridization\\
  Steps 1–9 are 10 minute incubations at the following temperatures:\\
  69;66;63;58;54;52;50;48;46\degree~C\\
  Step 10 is a 45\degree~C incubation for 1 hour\\
  Steps 11–14 are 10 minute incubations at the following templates:\\
  43;41;39;37\degree~C\\
  Final incubation is at 37\degree~C for $\infty$\\

\textbf{SeqZip ligation program}
ROY-37-4 | T4 Rnl2 RNA-template DNA:DNA ligation\\
  1. 37\degree~C for 18 hours\\
  2. 10\degree~C for $\infty$ \\

