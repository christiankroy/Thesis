% !TEX root = /Users/royc/Google_Drive/Thesis/RoyC_Umass_Thesis.tex
\lhead{\emph{Definitions}}
\label{hd:Definitions} \listDefinitions

\begin{description}
  \item[RNA-Seq]
  A technology wherein RNA is fragmented, converted to DNA, and analyzed on a high-throughput sequencing instrument

  \item[A ``Read'']
  The sequence of nucleotides produced from each spot on a high-throughput sequencing machine

  \item[Insert]
  The RNA molecule captured between two cloning sequences in a high-throughput sequencing library preparation workflow

  \item[Read length]
  The number of nucleotides for each given ``read''

  \item[Read depth]
  The number of reads obtained from each high-throughput sequencing analysis

  \item[Coverage]
  A measure of the number of times each nt of a genome is sequenced. E.g. 100 million reads of a 10 million nt genome = 10X coverage, assuming uniform distribution of the ``reads''

  \item[Paired-end]
  When both sides of a DNA insert or template are sequenced, utilizing the original length of DNA between the reads to facilitate mapping (\cite{Roach1995}).

  \item[Scaffold or Contig]
  A draft sequence of nucleotides, meant to represent the actual biological sequence as closely as possible, examples include unassembled fragments of chromosomes or fragments of mRNA transcripts.

  \item[Argonaute]
  Protein(s) belonging to a group containing a Piwi (P-element induced wimpy testes) domain, that bind nucleic acids and participate in many target-guided processes, including RNA Interference, and RNA-induced transcript/gene silencing.

  \item[Ligamer]
  A DNA oligonucleotide containing two distinct regions of complementarity to a 5\textprime~ and 3\textprime~ section of RNA. Each region is normalized for $^{Tm}$ such that the length of each section is \textasciitilde15--30 nt. These two regions are connected by a short sequence of the designer's choice, usually >5 nt in length. Each ligamers overall length is \textasciitilde45\textendash 60 nt. See figures \ref{SeqZipMethod:fig:Original SeqZip Diagram} and \ref{SeqZipPaper:fig:Roy2014 F1}.
  
  \end{description}
  \clearpage % Start a new page