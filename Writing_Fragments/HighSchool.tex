  \subsection{Why did I want to study AS?}

    Highschool Biology class is where I first learned about how ``genes'' - these things which I always heard everthing was ``in'' - actually worked.  There was a picture of a squiglly line called mRNA. This line was broken into boxes marked exons.  mRNA went on to code for protein, and along with a good part water, is what I was made of. Makes sense. A sublte note on the page drew my eye. It pointed to one of the exon boxes and said ``alternative splicing.'' What's this? The arrangement of the exons is not set in stone? It was around this time that the human genome was being finialized, we had soemthing like 3.9 billion basepairs, and this little box only contained ~150 bases. The implication struck me immediately. If these boxes could be alternatively arranged, then the possibilities of unique mRNA molecules is staggering! Far more then the 20K protein coding genes we are now known to possess. Something that also made sense - this \textit{MUST} be the reason why humans are so much more advanced then everything else - our mRNAs are alternatively spliced.

    That was ~10 years before I found myself developing a methodology that would assist in routine measurement of alternatively spliced mRNAs. Strange to think that I was amazed with a biological process I would go on to study. I was similarly amazed to learn about Sanger sequencing - it seemed so elegant to me. Another topic that truely interests me is HTS.