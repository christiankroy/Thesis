  \subsection{Function of long RNAs}

    We \textit{know} that some long RNAs are functional. What are these?

  \subsection{Chromatin regulation}

    We also know that some long RNAs regulation Chromatin structure. What are your feelings as to the importance of this fact?

  \subsection{PTGS}

    Long RNAs ability to do post transcriptional gene regulation, including piRNAs, and Xist, etc....

  \subsection{How are they important?}

    Conservation of these things is not obvious - if they are not conserved - are they important? Maybe talk about how MALAT1 is highly expressed, but seems to be dispensable.

  \subsection{What regulates their tissue-specific expression?}

    Do they important some of the special sauce that makes tissues different from one other, more so then the mRNAs changes which can be extreme, but not terribly so....

    What determines AS splicing decisions? It is not connectivity in splicing, and it seems to be SR and hnRNP proteins. Splicing is tissue specific. So is it the tissue-specific expression of SR and hnRNP proteins the is the main determinant of AS outcomes? How does chromatin state and organization play into AS decisions? 

  \subsection{Technological Improvements}

    How does one perform deep and broad analysis of mRNAs using second-generation HTS give the tremendous log-range over which they are expressed?

    This is the area of knowledge keeps many motivated to perform basic research every day. What secrets does the transcriptome have in store that we haven't even \textit{thought} about? Only through pushing the boundaries of the last two sections can we begin to think beyond the edge of map and formulate testable hypothesis. Here I propose a few outlandish ideas for Unknown Unknowns. 